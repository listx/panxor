\documentclass[letterpaper,twoside,12pt]{article}
\usepackage[vmargin=1in,hmargin=1in]{geometry}
\usepackage[titles]{tocloft}

% make XeTeX work properly with the \- (discretionary intraword hyphen),
% \textsuperscript, and \textsubscript commands (also loads the 'xunicode'
% package and other packages to make XeTeX work better; it also loads the
% 'fontspec' package)
\usepackage{xltxtra}

% import predefined colors
\usepackage[x11names]{xcolor}

% shortcut command for footnotes
\newcommand*\fn[1]{\footnote{#1}}

\usepackage{listings}
\lstloadlanguages{Haskell}
\lstnewenvironment{code}
	{\lstset{tabsize=4}%
		\csname lst@SetFirstLabel\endcsname}
		{\csname lst@SaveFirstLabel\endcsname}
	\lstset{
		basicstyle=\small\ttfamily,
		frame=lines,
		framexleftmargin=0.5em,
		framexrightmargin=0.5em,
		backgroundcolor=\color{LemonChiffon1},
		showstringspaces=false,
		flexiblecolumns=false,
    }

\renewcommand*\oldstylenums{\addfontfeature{Numbers={OldStyle}}}
\newcommand*\ct[1]{\texttt{\color{DeepPink3}#1}}
\setromanfont[Mapping=tex-text,Ligatures=Common]{Charis SIL}
\setsansfont[Mapping=tex-text,Ligatures=Common,Scale=0.8]{DejaVu Sans Condensed}
\setmonofont[Scale=0.8]{DejaVu Sans Mono}

% intra-document hyperlinks
\usepackage[colorlinks,linkcolor=blue]{hyperref}

\begin{document}
\title{The \texttt{panxor} Source Manual}
\author{Linus Arver}
\date{\today}
\maketitle
\tableofcontents
\section{Introduction}

\ct{panxor} applies a binary XOR operation to arbitrary numbers in hexadecimal, decimal, octal, and/or binary formats.

\documentclass[letterpaper,twoside,12pt]{article}
\usepackage[vmargin=1in,hmargin=1in]{geometry}
\usepackage[titles]{tocloft}

% make XeTeX work properly with the \- (discretionary intraword hyphen),
% \textsuperscript, and \textsubscript commands (also loads the 'xunicode'
% package and other packages to make XeTeX work better; it also loads the
% 'fontspec' package)
\usepackage{xltxtra}

% import predefined colors
\usepackage[x11names]{xcolor}

% shortcut command for footnotes
\newcommand*\fn[1]{\footnote{#1}}

\usepackage{listings}
\lstloadlanguages{Haskell}
\lstnewenvironment{code}
	{\lstset{tabsize=4}%
		\csname lst@SetFirstLabel\endcsname}
		{\csname lst@SaveFirstLabel\endcsname}
	\lstset{
		basicstyle=\small\ttfamily,
		frame=lines,
		framexleftmargin=0.5em,
		framexrightmargin=0.5em,
		backgroundcolor=\color{LemonChiffon1},
		showstringspaces=false,
		flexiblecolumns=false,
    }

\renewcommand*\oldstylenums{\addfontfeature{Numbers={OldStyle}}}
\newcommand*\ct[1]{\texttt{\color{DeepPink3}#1}}
\setromanfont[Mapping=tex-text,Ligatures=Common]{Charis SIL}
\setsansfont[Mapping=tex-text,Ligatures=Common,Scale=0.8]{DejaVu Sans Condensed}
\setmonofont[Scale=0.8]{DejaVu Sans Mono}

% intra-document hyperlinks
\usepackage[colorlinks,linkcolor=blue]{hyperref}

\begin{document}
\title{The \texttt{panxor} Source Manual}
\author{Linus Arver}
\date{\today}
\maketitle
\tableofcontents
\section{Introduction}

\ct{panxor} applies a binary XOR operation to arbitrary numbers in hexadecimal, decimal, octal, and/or binary formats.

\documentclass[letterpaper,twoside,12pt]{article}
\usepackage[vmargin=1in,hmargin=1in]{geometry}
\usepackage[titles]{tocloft}

% make XeTeX work properly with the \- (discretionary intraword hyphen),
% \textsuperscript, and \textsubscript commands (also loads the 'xunicode'
% package and other packages to make XeTeX work better; it also loads the
% 'fontspec' package)
\usepackage{xltxtra}

% import predefined colors
\usepackage[x11names]{xcolor}

% shortcut command for footnotes
\newcommand*\fn[1]{\footnote{#1}}

\usepackage{listings}
\lstloadlanguages{Haskell}
\lstnewenvironment{code}
	{\lstset{tabsize=4}%
		\csname lst@SetFirstLabel\endcsname}
		{\csname lst@SaveFirstLabel\endcsname}
	\lstset{
		basicstyle=\small\ttfamily,
		frame=lines,
		framexleftmargin=0.5em,
		framexrightmargin=0.5em,
		backgroundcolor=\color{LemonChiffon1},
		showstringspaces=false,
		flexiblecolumns=false,
    }

\renewcommand*\oldstylenums{\addfontfeature{Numbers={OldStyle}}}
\newcommand*\ct[1]{\texttt{\color{DeepPink3}#1}}
\setromanfont[Mapping=tex-text,Ligatures=Common]{Charis SIL}
\setsansfont[Mapping=tex-text,Ligatures=Common,Scale=0.8]{DejaVu Sans Condensed}
\setmonofont[Scale=0.8]{DejaVu Sans Mono}

% intra-document hyperlinks
\usepackage[colorlinks,linkcolor=blue]{hyperref}

\begin{document}
\title{The \texttt{panxor} Source Manual}
\author{Linus Arver}
\date{\today}
\maketitle
\tableofcontents
\section{Introduction}

\ct{panxor} applies a binary XOR operation to arbitrary numbers in hexadecimal, decimal, octal, and/or binary formats.

\documentclass[letterpaper,twoside,12pt]{article}
\usepackage[vmargin=1in,hmargin=1in]{geometry}
\usepackage[titles]{tocloft}

% make XeTeX work properly with the \- (discretionary intraword hyphen),
% \textsuperscript, and \textsubscript commands (also loads the 'xunicode'
% package and other packages to make XeTeX work better; it also loads the
% 'fontspec' package)
\usepackage{xltxtra}

% import predefined colors
\usepackage[x11names]{xcolor}

% shortcut command for footnotes
\newcommand*\fn[1]{\footnote{#1}}

\usepackage{listings}
\lstloadlanguages{Haskell}
\lstnewenvironment{code}
	{\lstset{tabsize=4}%
		\csname lst@SetFirstLabel\endcsname}
		{\csname lst@SaveFirstLabel\endcsname}
	\lstset{
		basicstyle=\small\ttfamily,
		frame=lines,
		framexleftmargin=0.5em,
		framexrightmargin=0.5em,
		backgroundcolor=\color{LemonChiffon1},
		showstringspaces=false,
		flexiblecolumns=false,
    }

\renewcommand*\oldstylenums{\addfontfeature{Numbers={OldStyle}}}
\newcommand*\ct[1]{\texttt{\color{DeepPink3}#1}}
\setromanfont[Mapping=tex-text,Ligatures=Common]{Charis SIL}
\setsansfont[Mapping=tex-text,Ligatures=Common,Scale=0.8]{DejaVu Sans Condensed}
\setmonofont[Scale=0.8]{DejaVu Sans Mono}

% intra-document hyperlinks
\usepackage[colorlinks,linkcolor=blue]{hyperref}

\begin{document}
\title{The \texttt{panxor} Source Manual}
\author{Linus Arver}
\date{\today}
\maketitle
\tableofcontents
\input{intro.tex}
\input{../src/panxor.lhs}
\end{document}

\end{document}

\end{document}

\input{../src/PANXOR/Core.lhs}
\input{../src/PANXOR/Option.lhs}
\input{../src/PANXOR/Meta.lhs}
\input{../src/PANXOR/Util.lhs}
\end{document}
